\documentclass{article}
\usepackage{graphicx}
\usepackage{color}
\usepackage{comment}
\usepackage{amssymb}
\usepackage{amsthm}

\newtheorem{definition}{Definition}
\newtheorem{property}{Property}
\newtheorem{proposition}{Proposition}
\newtheorem{conjecture}{Conjecture}
\newtheorem{example}{Example}
\newtheorem{notation}{Notation}
\newtheorem{theorem}{Theorem}

%%%%%%%%%%%%%%%%%%%%%%%%%%%%%%%%%%%%%%%%%%%%%%%%%%%%%%%%%%%%%%%%%%%%%

% Algorithms and pseudo code
\usepackage{verbatim}
\usepackage{algorithm}
\usepackage{algorithmicx}
\usepackage{algpseudocode}

% Graphics and display
\usepackage{float}
\usepackage{graphicx}
\usepackage{subfig}
\usepackage{enumerate}
\usepackage{url}
\usepackage{multirow}

% Math symbols and environments
\usepackage{amsmath}
\usepackage{amssymb}
\usepackage{stmaryrd} % \varcurlyvee


\usepackage{calc}%#%
\usepackage{enumitem} %%% used in framed graph

\date{ }
% TikZ

\usepackage{tikz}
\usetikzlibrary{shapes,arrows}
\usetikzlibrary{positioning}
\usetikzlibrary{calc}

\usepackage{tikz-3dplot}

\definecolor{myyellow}{RGB}{255,255,150}
\definecolor{mylavender}{RGB}{125,249,255}
\definecolor{mygreen}{RGB}{144,238,14}
\definecolor{myred}{RGB}{255,0,0}

\newcommand\mytext[3][\scriptsize]{#2\\#1 #3}
\newcommand\mynode[4][]{%
  \node[mynode,#1,text width=\the\dimexpr#2cm] (#3) {\mytext{#3}{#4}}; 
}
\newcommand\mynot[4][]{%
  \node[mynot,#1,text width=\the\dimexpr#2cm] (#3) {\mytext{#3}{#4}}; 
}

\setcounter{secnumdepth}{4}

%%%%%%%%%%%%%%%%
%%%% Macros %%%%
%%%%%%%%%%%%%%%%

%%% Math
\newcommand{\nat}{\mathbb{N}}   % Natural numbers
\newcommand{\rat}{\mathbb{Q}}   % Rational numbers
\newcommand{\real}{\mathbb{R}}  % Real numbers
\newcommand{\runit}{[0, 1]}    % The real unit interval


% = with "hip." on the top, useful for indicating where a hypothesis comes in
\newcommand{\heq}{\stackrel{\text{\fontsize{3pt}{3pt}\selectfont hip.}}{=}}
% = because of the de Morgan laws.
\newcommand{\dmeq}{\stackrel{\text{\tiny{dM}}}{=}}


%%% Sets
\newcommand{\args}{\mathcal{A}} % Set of all arguments
\newcommand{\att}{\mathcal{R}}  % Set of all attacks
\newcommand{\valueset}{L}

%%% Votes on arguments
\newcommand{\varg}{V_{\args}}   % Function giving votes on arguments
\newcommand{\vargpro}[1]{\varg^+\left(#1\right)} % Pro votes on arguments
\newcommand{\vargcon}[1]{\varg^-\left(#1\right)} % Con votes on arguments
\newcommand{\vargtot}[1]{\varg^{max}\left(#1\right)} % Max votes on arguments

%%% Votes on attacks
\newcommand{\vatt}{V_{\att}}   % Function giving votes on attacks
\newcommand{\vattpro}[1]{\vatt^+\left(#1\right)} % Pro votes on attacks
\newcommand{\vattcon}[1]{\vatt^-\left(#1\right)} % Con votes on attacks

%%% Attack relations
% Attackers of a given argument
\newcommand{\attackers}[1]{\att^\text{-}\left(#1\right)} 
% Attackers of a given argument for the alternative framework F'
\newcommand{\altattackers}[1]{\att^{\prime\text{-}}\left(#1\right)}
% Ancestors of given argument according to the attack relation
\newcommand{\ancestors}[1]{\att^*\left(#1\right)} 

%%% Frameworks
\newcommand{\safid}{F}               % A single SAF, given by identifier
\newcommand{\safset}{\mathcal{F}}    % Set of all SAFs

\newcommand{\saf}{\safid = \safbody} % Framework id and respective tuple
\newcommand{\safbody}{\langle \args, \att, \varg, \vatt \rangle} % SAF tuple
\newcommand{\oldsaf}{\safid = \oldsafbody} % Ex Framework id and respective tuple
\newcommand{\oldsafbody}{\langle \args, \att, V \rangle} % old SAF tuple
% Alternative framework, same as \safbody but with ' everywhere ;)
\newcommand{\altsafbody}{\langle \args', \att', \varg', \vatt' \rangle} 

%%% Semantics
\newcommand{\semid}{\mathcal{S}}        % Semantic framework identifier
% Semantic framework tuple
\newcommand{\sembody}{\left\langle \valueset,\SAFand_1, \SAFand_2,\SAFor,\lnot,\tau \right\rangle}
\newcommand{\semdef}{\semid = \sembody}     % Semantic framework id and tuple
\newcommand{\semprod}[1]{\semid^\cdot_{#1}} % Product semantic framework
\newcommand{\semsub}{\semid^\text{-}}       % Subtraction semantic framework
\newcommand{\semmax}{\semid^\text{max}}     % Max semantic framework

\newcommand{\sembodyNew}{\left\langle \valueset,\SAFand_\mathcal{A}, \SAFand_\mathcal{R}, \SAFor, \lnot, \tau^{\args}, \tau^{\att}\right\rangle} %New semantic body

\newcommand{\SAFand}{\curlywedge}     % Logical and for SAF equations 
\newcommand{\SAFor}{\curlyvee}        % Logical or for SAF equations
\DeclareMathOperator*{\SAFOr}{\bigcurlyvee} % Big or notation, works as \sum
                             %\varcurlyvee also works, but is smaller
\DeclareMathOperator*{\SAFAnd}{\bigcurlywedge} % Big and notation, works as \sum

\newcommand{\modelset}{\mathcal{M}}   % Set of all models


%#% old commands
\newcommand{\afit}{\textit{AF}}
\newcommand{\af}{\afit = \langle \args, \att \rangle}
\newcommand{\vote}{V}
\newcommand{\sem}{\mathcal{S}}

\newcommand{\ssv}{\mathcal{V}}
\newcommand{\tv}{\mathcal{T}}
\newcommand{\pv}{\mathcal{P}}
\newcommand{\xv}{\mathcal{X}}
\newcommand{\ev}{\mathcal{E}}

\newcommand{\safit}{F}

\newcommand{\tupd}{\curlywedge}
\newcommand{\tatt}{\curlyvee}
\newcommand{\Tatt}{\varcurlyvee}

\newcommand{\argarray}{\{x_1, ..., x_n\}}

\newcommand{\voteset}{\mathcal{V}}
\newcommand{\vpro}{\vote^+}
\newcommand{\vcon}{\vote^-}

%%% Mappings
\newcommand{\mapping}{\Phi}

%%% Macro for framed graph
\newlist{tikzitem}{itemize}{1}
\setlist[tikzitem,1]{label=$\bullet$,nolistsep,leftmargin=*}

%%%%%%%%%%%%%%%%%%%%%%%%%%%%%%%%%%%%%%%%%%%%%%%%%%%%%%%%%%%%%%%%%%%%%
%%%%%%%%%%%%%%%%%%%%%%%%%%%%%%%%%%%%%%%%%%%%%%%%%%%%%%%%%%%%%%%%%%%%%

\begin{document}

%\title{Report regarding the ongoing research}
%\maketitle

%\section{Main}


\begin{definition}[Extended social argumentation frameworks]
An \emph{extended social argumentation framework} is a 4-tuple $\saf$, where
\begin{itemize}
  \item $\args$ is the set of arguments,
  \item $\att \subseteq \args \times \args$ is a binary attack relation between arguments,
  \item $\varg : \args \to \nat \times \nat$ stores the crowd's pro and con votes for each argument.
  \item $\vatt : \att \to \nat \times \nat$ stores the crowd's pro and con votes for each attack.
\end{itemize}
\end{definition}

\begin{definition}
\label{def:voteAgg}
[Argument Vote Aggregation Function]
Let $\args$ be a set of arguments, $\varg : \args \to \nat \times \nat$ a function that stores the crowd's pro and con votes for each argument and  $\valueset$ a totally ordered set with top and bottom elements $\top$, $\bot$.

We say that an argument vote aggregation function $\tau^{\args}$ is any function such that $\tau^{\args}: \args \times {2}^{\args} \to L$.
\end{definition}

\begin{definition}
\label{def:voteAgg}
[Attack Vote Aggregation Function]
Let $\att$ be a set of attack relations, $\vatt : \att \to \nat \times \nat$ a function that stores the crowd's pro and con votes for each attack and  $\valueset$ a totally ordered set with top and bottom elements $\top$, $\bot$.

We say that an attack vote aggregation function $\tau^{\att}$ is any function such that $\tau^{\att}: \att \times {2}^{\att} \to L$.
\end{definition}

\begin{definition}[Semantic Framework]
\label{def:semfram}
A semantic framework is a 6-tuple \\$\sembodyNew$ where:

\begin{itemize}
  \item $\valueset$ is a totally ordered set with top and bottom elements $\top$, 
$\bot$, containing all possible valuations of an argument. 

  \item $\SAFand_\args,\SAFand_\att:\valueset\times \valueset\rightarrow \valueset$, are two binary algebraic operations used to restrict strengths to given values.
  
  \item $\SAFor:\valueset\times \valueset\rightarrow \valueset$, is a binary algebraic operation on argument valuations used to combine or aggregate valuations and strengths.
  
  \item $\lnot:\valueset\rightarrow \valueset$ is a unary algebraic operation for computing a restricting value corresponding to a given valuation or strength.
  
  \item $\tau^{\args}$, $\tau^{\att}$ are two vote aggregation functions for computing a social support value of an argument and an attack relation respectively, given some context.

\end{itemize}
\end{definition}


\begin{notation}
Let $\saf$ be an ESAF, $\sem = \sembodyNew$ a semantic framework. Then, let
\begin{itemize}
\item $\attackers{a} \triangleq \{a_i \in \args: (a_i, a) \in \att\}$ be the set of direct attackers of an argument $a \in \args$, 
\item $v^r: \nat \times \nat \to \real$ be a function s.t. $v^r(x, y) \triangleq \frac{x}{y}$ where $x, y \in \mathbb{N}$,
\item $v^t: \nat \times \nat \to \nat$ be a function s.t. $v^t(x, y) \triangleq x + y$ where $x, y \in \mathbb{N}$, 
\item 
\begin{itemize}
\item $\vargpro a \triangleq x$ denote the number of positive votes for argument $a$,
\item $\vargcon a \triangleq y$ denote the number of negative votes for argument $a$,
\end{itemize}
whenever $\varg (a) = (x, y)$,
\\
\item $\varg: 2^\args \to 2^{\nat \times \nat}$ be a function s.t. $\varg(\mathcal{A}^{'}) = \{\varg(a)$ $|$ $a \in \mathcal{A^{'}}\}$,
%\item $\tau(a, A^{'}) \triangleq \tau(V_{\mathcal{A}}(a), \varg(A^{'}))$ denote the social support for an argument $a$, within a set of arguments $A^{'}$, via utilizing a vote aggregation function $\tau$,
\item
\begin{itemize}
\item $\vattpro{(a, b)} \triangleq x$ denote the number of positive votes for an attack relation between arguments  $a$ and $b$,
\item  $ \vattcon{(a, b)} \triangleq y$ denote the number of negative votes for an attack relation between arguments  $a$ and $b$,
\end{itemize}
whenever $\vatt ((a, b)) = (x, y)$,
\\
\item $\vatt: 2^\att \to 2^{\nat \times \nat}$ be a function s.t. $\vatt(\mathcal{R}^{'}) = \{\vatt((a, b))$ $|$ $(a, b) \in \mathcal{R}^{'}\}$,
%\item $\tau((a, b), R^{'}) \triangleq \tau(V_{\mathcal{R}}((a, b)), \vatt(R^{'}))$ denote the social support for an attack relation between arguments  $a$ and $b$, within a set of attack relations $R^{'}$, via utilizing a vote aggregation function $\tau$,
\item$$\SAFOr_{x \in X} x \triangleq\left(\left(\left(  x_{1}\SAFor x_{2}\right) \SAFor...\right)\SAFor x_{n}\right)$$ $X=\left\{  x_{1},x_{2},...,x_{n}\right\}$ denote the aggregation of a multiset of elements of $\valueset$. 

\item $max: 2^{\nat} \to \nat$ be a function s.t. it returns the maximum value amongst the natural numbers from the non-empty set given as the input.

\item $\omega:  2^{\nat \times \nat} \to 2^{\nat}$ be a function s.t. $\omega(\mathcal{X}) = \{a_{1}+a_{2}$ $|$ $(a_{1}, a_{2}) \in \mathcal{X}\}$,

\end{itemize}
\end{notation}

\begin{definition}[Model] 
\label{def:model}
  Let $\saf$ be a social argumentation framework, $\sem = \sembodyNew$ be a semantic framework. An $\semid$-model of $\safid$ is a total mapping $M : \args \rightarrow \valueset$ such that for all $a \in \args$,
  $$\displaystyle M(a) = \tau^\args(a, \args ) \SAFand_\args \lnot \SAFOr_{a_i \in \attackers{a}} \left(\tau^\att\left((a_i, a), \att \right) \SAFand_\att M\left(a_i\right)\right)$$
\end{definition}


\begin{definition}
\label{def:enhVoteAgg}
[Enhanced Vote Aggregation]
 Given $\varg : \args \to \nat \times \nat$ and $\varg: 2^\args \to 2^{\nat \times \nat}$, enhanced vote aggregation function
$\tau^\args_{e}: \args \times {2}^{\args} \rightarrow\lbrack0,1]$ is the argument vote aggregation function such that
\[
\tau^\args_{e}  (a, \mathcal{A}^{'})  = \left\{
\begin{array}
[c]{lll}
0 &  & \forall x \in \varg (\mathcal{A}^{'}),\hspace{1mm} x = (0,0)\\
\frac{\vargpro a}{\vargpro a + \vargcon a+\frac{1}{max(\omega(\varg (\mathcal{A}^{'})))}} &  & \text{otherwise}%
\end{array}
\right.
\]
%\tau_{e}\left(  v^{+},v^{-}, v_{max}\right)  =
%where $v_{max}$ stands for the maximum number of votes for an argument in the system.
\end{definition}


\newpage

{\color{teal}
 
\emph{Personal Comments:} 
-  Perhaps this still doesn't look simple enough. Changing to the new $\tau$ definition has saved us from introducing the old '\emph{abuse of notations}', however we can't adopt the same aggregation function for votes and attacks anymore since the function domains differ. (I haven't written it down above but if we are to adopt this notation, we shall also introduce a concrete aggregation function for the attack relations, in-line with the \emph{enhanced vote aggregation function}).
\vspace{5 mm}

- I assume the denominator of the function in Def5 looks bloated, especially in this version. In a nutshell, we need the following mechanisms: mapping each argument to their respective votes, aggregating each pos/con vote pair  and computing the maximum of those values.  I will continue thinking about it, unfortunately up to now I couldn't come up with a more compact notation.
\vspace{5 mm}

- The old notation(\emph{repJuice.pdf}) reads better in this aspect, since it skips the first step(going from arguments to pos/con values) as the last argument of the aggregation function is pairs of naturals. On a related note, last Friday you criticized me for using four function definitions for this notion. I guess I couldn't defend my stance properly so please let me have another go at it. 

Firstly since you allowed me to define $max$ implicitly, $bimax$ is gone. Secondly the $\omega$ function(last week's $agg$) used to utilize $v^{t}$, specifically it was as follows: $\omega(\mathcal{X}) = \{v^t(a_1, a_2)$ $|$ $(a_1, a_2) \in \mathcal{X}\}$

If you may recall the reason why we defined functions $v^t$ and $v^r$ in the first place was not in the limits of this document. But it was because in the upcoming properties, the corresponding terms were occurring a handful times and thus you requested me to include these two functions in the definitions. So surely it may seem unnecessary using $v^t$ in the setting so far, but my thinking was that with the upcoming material it would make more sense that way.

All in all, as can be seen in both .pdf files current $\omega$ definition omits using $v^t$: $\omega(\mathcal{X}) = \{v^t(a_1, a_2)$ $|$ $(a_1, a_2) \in \mathcal{X}\}$. And consequently the \emph{enhanced vote aggregation function} definition in \emph{repJuice.pdf} only utilizes two external functions: the $max$ and $\omega$.

}
\end{document}