\documentclass{article}
\usepackage{graphicx}
\usepackage{color}
\usepackage{comment}
\usepackage{amssymb}
\usepackage{amsthm}

\newtheorem{definition}{Definition}
\newtheorem{property}{Property}
\newtheorem{proposition}{Proposition}
\newtheorem{conjecture}{Conjecture}
\newtheorem{example}{Example}
\newtheorem{notation}{Notation}
\newtheorem{theorem}{Theorem}

%%%%%%%%%%%%%%%%%%%%%%%%%%%%%%%%%%%%%%%%%%%%%%%%%%%%%%%%%%%%%%%%%%%%%

% Algorithms and pseudo code
\usepackage{verbatim}
\usepackage{algorithm}
\usepackage{algorithmicx}
\usepackage{algpseudocode}

% Graphics and display
\usepackage{float}
\usepackage{graphicx}
\usepackage{subfig}
\usepackage{enumerate}
\usepackage{url}
\usepackage{multirow}

% Math symbols and environments
\usepackage{amsmath}
\usepackage{amssymb}
\usepackage{stmaryrd} % \varcurlyvee


\usepackage{calc}%#%
\usepackage{enumitem} %%% used in framed graph

\date{ }
% TikZ

\usepackage{tikz}
\usetikzlibrary{shapes,arrows}
\usetikzlibrary{positioning}
\usetikzlibrary{calc}

\usepackage{tikz-3dplot}

\definecolor{myyellow}{RGB}{255,255,150}
\definecolor{mylavender}{RGB}{125,249,255}
\definecolor{mygreen}{RGB}{144,238,14}
\definecolor{myred}{RGB}{255,0,0}

\newcommand\mytext[3][\scriptsize]{#2\\#1 #3}
\newcommand\mynode[4][]{%
  \node[mynode,#1,text width=\the\dimexpr#2cm] (#3) {\mytext{#3}{#4}}; 
}
\newcommand\mynot[4][]{%
  \node[mynot,#1,text width=\the\dimexpr#2cm] (#3) {\mytext{#3}{#4}}; 
}

\setcounter{secnumdepth}{4}

%%%%%%%%%%%%%%%%
%%%% Macros %%%%
%%%%%%%%%%%%%%%%

%%% Math
\newcommand{\nat}{\mathbb{N}}   % Natural numbers
\newcommand{\rat}{\mathbb{Q}}   % Rational numbers
\newcommand{\real}{\mathbb{R}}  % Real numbers
\newcommand{\runit}{[0, 1]}    % The real unit interval


% = with "hip." on the top, useful for indicating where a hypothesis comes in
\newcommand{\heq}{\stackrel{\text{\fontsize{3pt}{3pt}\selectfont hip.}}{=}}
% = because of the de Morgan laws.
\newcommand{\dmeq}{\stackrel{\text{\tiny{dM}}}{=}}


%%% Sets
\newcommand{\args}{\mathcal{A}} % Set of all arguments
\newcommand{\att}{\mathcal{R}}  % Set of all attacks
\newcommand{\valueset}{L}

\newcommand{\obj}{\mathcal{O}} % Set of all arguments and attacks as a union

%%% Votes on arguments
\newcommand{\varg}{V_{\args}}   % Function giving votes on arguments
\newcommand{\vargpro}[1]{\varg^+\left(#1\right)} % Pro votes on arguments
\newcommand{\vargcon}[1]{\varg^-\left(#1\right)} % Con votes on arguments
\newcommand{\vargtot}[1]{\varg^{max}\left(#1\right)} % Max votes on arguments

%%% Votes on attacks
\newcommand{\vatt}{V_{\att}}   % Function giving votes on attacks
\newcommand{\vattpro}[1]{\vatt^+\left(#1\right)} % Pro votes on attacks
\newcommand{\vattcon}[1]{\vatt^-\left(#1\right)} % Con votes on attacks

%%% Attack relations
% Attackers of a given argument
\newcommand{\attackers}[1]{\att^\text{-}\left(#1\right)} 
% Attackers of a given argument for the alternative framework F'
\newcommand{\altattackers}[1]{\att^{\prime\text{-}}\left(#1\right)}
% Ancestors of given argument according to the attack relation
\newcommand{\ancestors}[1]{\att^*\left(#1\right)} 

%%% Frameworks
\newcommand{\safid}{F}               % A single SAF, given by identifier
\newcommand{\safset}{\mathcal{F}}    % Set of all SAFs

\newcommand{\saf}{\safid = \safbody} % Framework id and respective tuple
\newcommand{\safbody}{\langle \args, \att, \varg, \vatt \rangle} % SAF tuple
\newcommand{\oldsaf}{\safid = \oldsafbody} % Ex Framework id and respective tuple
\newcommand{\oldsafbody}{\langle \args, \att, V \rangle} % old SAF tuple
% Alternative framework, same as \safbody but with ' everywhere ;)
\newcommand{\altsafbody}{\langle \args', \att', \varg', \vatt' \rangle} 

\newcommand{\safbodyO}{\langle \args, \att, \obj, V \rangle} % SAF tuple with objects
\newcommand{\safO}{\safid = \safbodyO} % Framework id and respective tuple with objects

%%% Semantics
\newcommand{\semid}{\mathcal{S}}        % Semantic framework identifier
% Semantic framework tuple
\newcommand{\sembody}{\left\langle \valueset,\SAFand_1, \SAFand_2,\SAFor,\lnot,\tau \right\rangle}
\newcommand{\semdef}{\semid = \sembody}     % Semantic framework id and tuple
\newcommand{\semprod}[1]{\semid^\cdot_{#1}} % Product semantic framework
\newcommand{\semsub}{\semid^\text{-}}       % Subtraction semantic framework
\newcommand{\semmax}{\semid^\text{max}}     % Max semantic framework

\newcommand{\sembodyNew}{\left\langle \valueset,\SAFand_\mathcal{A}, \SAFand_\mathcal{R},\SAFor,\lnot,\tau \right\rangle} %New semantic body
\newcommand{\sembodyNewB}{\left\langle \valueset,\SAFand_\mathcal{A}, \SAFand_\mathcal{R},\SAFor,\lnot,\tau_{e} \right\rangle} %New semantic body

\newcommand{\SAFand}{\curlywedge}     % Logical and for SAF equations 
\newcommand{\SAFor}{\curlyvee}        % Logical or for SAF equations
\DeclareMathOperator*{\SAFOr}{\bigcurlyvee} % Big or notation, works as \sum
                             %\varcurlyvee also works, but is smaller
\DeclareMathOperator*{\SAFAnd}{\bigcurlywedge} % Big and notation, works as \sum

\newcommand{\modelset}{\mathcal{M}}   % Set of all models


%#% old commands
\newcommand{\afit}{\textit{AF}}
\newcommand{\af}{\afit = \langle \args, \att \rangle}
\newcommand{\vote}{V}
\newcommand{\sem}{\mathcal{S}}

\newcommand{\ssv}{\mathcal{V}}
\newcommand{\tv}{\mathcal{T}}
\newcommand{\pv}{\mathcal{P}}
\newcommand{\xv}{X}
\newcommand{\ev}{\mathcal{E}}

\newcommand{\safit}{F}

\newcommand{\tupd}{\curlywedge}
\newcommand{\tatt}{\curlyvee}
\newcommand{\Tatt}{\varcurlyvee}

\newcommand{\argarray}{\{x_1, ..., x_n\}}

\newcommand{\voteset}{\mathcal{V}}
\newcommand{\vpro}{\vote^+}
\newcommand{\vcon}{\vote^-}

%%% Mappings
\newcommand{\mapping}{\Phi}

%%% Macro for framed graph
\newlist{tikzitem}{itemize}{1}
\setlist[tikzitem,1]{label=$\bullet$,nolistsep,leftmargin=*}

%%%%%%%%%%%%%%%%%%%%%%%%%%%%%%%%%%%%%%%%%%%%%%%%%%%%%%%%%%%%%%%%%%%%%
%%%%%%%%%%%%%%%%%%%%%%%%%%%%%%%%%%%%%%%%%%%%%%%%%%%%%%%%%%%%%%%%%%%%%

\begin{document}

%\title{Report regarding the ongoing research}
%\maketitle

%\section{Main}


\begin{definition}[Extended social argumentation frameworks]
An \emph{extended social argumentation framework} is a 4-tuple $\safO$, where
\begin{itemize}
  \item $\args$ is the set of arguments,
  \item $\att \subseteq \args \times \args$ is a binary attack relation between arguments,
  \item $\obj = \args \cup \att$ is the set of objects, composed by the union of the sets of arguments and attack relations,
  \item $V : \obj \to \nat \times \nat$ stores the crowd's pro and con votes for each object.
\end{itemize}
\end{definition}

\begin{definition}
\label{def:voteAgg}
[Vote Aggregation Function]
Given a totally ordered set $\valueset$ with top and bottom elements $\top$, $\bot$, a voting function $V$ and a set of objects $\obj$, a vote aggregation function $\tau$ is any function such that $\tau: \obj  \times {2}^{\obj} \to L$.
\end{definition}



\begin{definition}[Semantic Framework]
\label{def:semfram}
A semantic framework is a 6-tuple \\$\sembodyNew$ where:

\begin{itemize}
  \item $\valueset$ is a totally ordered set with top and bottom elements $\top$, 
$\bot$, containing all possible valuations of an argument. 

  \item $\SAFand_\args,\SAFand_\att:\valueset\times \valueset\rightarrow \valueset$, are two binary algebraic operations used to restrict strengths to given values.
  
  \item $\SAFor:\valueset\times \valueset\rightarrow \valueset$, is a binary algebraic operation on argument valuations used to combine or aggregate valuations and strengths.
  
  \item $\lnot:\valueset\rightarrow \valueset$ is a unary algebraic operation for computing a restricting value corresponding to a given valuation or strength.
  
%  \item $\tau$ is a vote aggregation function which given some context, aggregates positive and negative votes into a social support value.

  \item $\tau$ is a vote aggregation function which, given the votes, determines the social support of an object within a set of objects.

\end{itemize}
\end{definition}

\begin{notation}
Let $\safO$ be an ESAF, $\sem = \sembodyNew$ a semantic framework. Then, let
\begin{itemize}
\item
\begin{itemize}
\item $V^{+} (o) \triangleq x$ denote the number of positive votes for object $o$,
\item $V^{-} (o) \triangleq y$ denote the number of negative votes for object $o$,
\item $v^r: \obj \to \real$ be a function s.t. $v^r(o) \triangleq \frac{x}{x + y}$,
\item $v^t: \obj \to \nat$ be a function s.t. $v^t(o) \triangleq x + y$,  
\end{itemize}
whenever $V(o) = (x, y)$,
\\
\item $V: 2^\obj \to 2^{\nat \times \nat}$ be a function s.t. $V(\mathcal{O}^{'}) = \{V(o)$ $|$ $o \in \mathcal{O}^{'}\}$,
\item $V^{t}: 2^\obj \to 2^{\nat \times \nat}$ be a function s.t. $V^t(\mathcal{O}^{'}) = \{V^t(o)$ $|$ $o \in \mathcal{O}^{'}\}$,
\item $max: 2^{\nat} \to \nat$ be a function s.t. it returns the maximum value amongst the natural numbers from the non-empty multiset given as the input.
\item $\attackers{a} \triangleq \{a_i \in \args: (a_i, a) \in \att\}$ be the set of direct attackers of an argument $a \in \args$, 
\item$$\SAFOr_{x \in X} x \triangleq\left(\left(\left(  x_{1}\SAFor x_{2}\right) \SAFor...\right)\SAFor x_{n}\right)$$ $X=\left\{  x_{1},x_{2},...,x_{n}\right\}$ denote the aggregation of a multiset of elements of $\valueset$. 
\end{itemize}
\end{notation}

\begin{definition}[Model] 
\label{def:model}
  Let $\safO$ be a social argumentation framework, $\sem = \sembodyNew$ be a semantic framework. An $\semid$-model of $\safid$ is a total mapping $M : \args \rightarrow \valueset$ such that for all $a \in \args$,
  $$\displaystyle M(a) = \tau(a, \args ) \SAFand_\args \lnot \SAFOr_{a_i \in \attackers{a}} \left(\tau\left((a_i, a), \att \right) \SAFand_\att M\left(a_i\right)\right)$$
\end{definition}


\begin{definition}
\label{def:enhVoteAgg}
[Enhanced Vote Aggregation]
\\ Given a voting function $V$ and a set of objects $\obj$, enhanced vote aggregation function
$\tau_{e}:\obj  \times {2}^{\obj} \rightarrow\lbrack0,1]$ is the vote aggregation function such that
\[
\tau_{e}  (o, \mathcal{O})  = \left\{
\begin{array}
[c]{lll}
0 &  & V(o) = (0,0)\\
\frac{v^{+}(o)}{v^{t}(o)+\frac{1}{max(v^{t}(o \cup \mathcal{O}))}} &  & \text{otherwise}%
\end{array}
\right.
\]
\end{definition}


\begin{property}
\label{P1} [Absolute argument freeness] \\
Let $\tau$ be a a vote aggregation function given a set of values $\valueset$, a set of objects $\obj$ and a value function $V$. We say that $\tau$ is 'absolute argument free' if
% $X$ be a set of pairs of integers. %% <== sozlu hali
\begin{center}
$\forall o \in \obj$, $\tau (o, \obj) \neq \top$.
\end{center}
\end{property}

%The gist of the property can be captured in a verbal context with the following sentence: \emph{No argument enjoys perfect Social Support.}  \\


\begin{property}
\label{P2}[Precedence of the vote ratio]\\
Let $\tau$ be a a vote aggregation function given a set of values $\valueset$, a set of objects $\obj$ and a value function $V$. We say that $\tau$ is 'vote ratio precedent' if
\begin{center}
$\forall o_1, o_2 \in \obj$,  \\
$\left( v^{r}(o_{1}) \ge v^{r}(o_{2}) \right) \implies \left( \tau(o_{1}, \obj) \ge \tau(o_{2},\obj) \right)$. \\


\end{center}

\end{property}

%The gist of the property can be captured in a verbal context with the following sentence: \emph{If the value of the ratio of positive votes to the negative votes is higher for an argument $a$ than an argument $b$; $a$'s social support value exceeds the social support of $b$.}  \\



\begin{property}
\label{P3}[Precedence of the total number of votes] \\
Let $\tau$ be a a vote aggregation function given a set of values $\valueset$, a set of objects $\obj$ and a value function $V$. We say that $\tau$ is 'total votes precedent' if
\begin{center}
 $\forall o_1, o_2 \in \obj$,  \\
$\left( \left(v^{r}(o_{1}) = v^{r}(o_{2}) \right) \land  \left( v^{t}(o_{1}) > ( v^{t}(o_{2}) \right) \right) \implies \left( \tau(o_{1}, \obj) > \tau(o_{2}, \obj) \right)$. 
\\

\end{center}
\end{property}

%The gist of the property can be captured in a verbal context with the following sentence: \emph{When the ratios are equal, the function should return a higher social support value for the one with the higher number of total votes.}  \\





\begin{proposition}
Enhanced Vote Aggregation function respects Property \ref{P1}.
\end{proposition}

\begin{proof} %Trivial.\\
Let  $\valueset$ be a totally ordered set with top and bottom elements $\top$, $\bot$, $V$ a voting function, $\obj$ a set of objects and an arbitrary $o \in \obj$.

We consider the two main cases for the enhanced vote aggregation function with respect to $o$.

Firstly, if $V(o) = (0,0)$, by definition $\tau_{e}(o, \obj) = 0 < 1$.

Now assume $V(o) \neq (0,0)$. We know that for an arbitrary multiset of naturals $X$ where $\exists (n_1,n_2) \in X$ s.t. $(n_1,n_2) \neq (0,0)$, $max(X)  \in \nat^+$. And consequently $\frac{1}{max(X)} > 0$. Furthermore we have $v^{t}(o) = v^{+}(o) + v^{-}(o)$.

From the two inferences we get: $(v^{t}(o)+\frac{1}{max(v^{t}(o \cup \mathcal{O}))}) > v^{+}$.

% the denominator of the term in the social support function is simply the sum of the numerator and some $r \in {R}^+ $, the numerator is strictly less than the denominator. 
Hence it follows that: \\
\begin{center}
$ \tau_{e}(o,  \obj) = \frac{v^{+}(o)}{v^{t}(o)+\frac{1}{max(v^{t}(o \cup \mathcal{O}))}} < 1$.
\end{center}
\end{proof}

\begin{conjecture}
Enhanced Vote Aggregation function respects Property \ref{P2}.
\end{conjecture}

\begin{proposition}
Vote Aggregation function\cite{eml2013esaf} does not respect Property \ref{P2}.
\end{proposition}

\begin{proof}
It's sufficient to display a counter example.

Let  $\valueset$ be a totally ordered set with top and bottom elements $\top$, $\bot$, $V$ a voting function, $\obj$ a set of objects and $o_1, o_2 \in \obj$.

EXTEND PART

Furthermore assume $\epsilon = 0.1, V(o_1) = (1, 1)$ and $V(o_2) = (499, 501).$ 
%v_1^{+} = 1, v_1^{-} = 1, v_2^{+} = 499, v_2^{-} = 501$.

Since $v^{r}(o_{1}) \ge v^{r}(o_{2})$ via the property the following should hold $\tau(o_{1}, \obj) \ge \tau(o_{2},\obj)$.

However $\tau(v_1^{+}, v_1^{-}) = \tau(1, 1) \cong 0.4761$ and  $\tau(v_2^{+}, v_2^{-}) = \tau(499, 501) \cong 0.4989$.
Hence, $\tau(v_1^{+}, v_1^{-}) < \tau(v_2^{+}, v_2^{-})$.

We conclude a contradiction.
\end{proof}

\begin{conjecture}
Enhanced Vote Aggregation function respects Property \ref{P3}.
\end{conjecture}

\bibliographystyle{plain}
\bibliography{repBib}

%%
\begin{comment}
{\color{teal}
 
\emph{Pointers for discussion:} 
\begin{itemize}
\item
\end{itemize}
}
\end{comment}
%%
\end{document}